%%%%%%%%%%%%%%%%%%%%%%%%%%%%%%%%%%%%%%%%%
%
% CMPT 435
% Lab Three
%
%%%%%%%%%%%%%%%%%%%%%%%%%%%%%%%%%%%%%%%%%

%%%%%%%%%%%%%%%%%%%%%%%%%%%%%%%%%%%%%%%%%
% Short Sectioned Assignment
% LaTeX Template
% Version 1.0 (5/5/12)
%
% This template has been downloaded from: http://www.LaTeXTemplates.com
% Original author: % Frits Wenneker (http://www.howtotex.com)
% License: CC BY-NC-SA 3.0 (http://creativecommons.org/licenses/by-nc-sa/3.0/)
% Modified by Alan G. Labouseur  - alan@labouseur.com, and Ryan Munger - ryan.munger1@marist.edu
%
%%%%%%%%%%%%%%%%%%%%%%%%%%%%%%%%%%%%%%%%%

%----------------------------------------------------------------------------------------
%	PACKAGES AND OTHER DOCUMENT CONFIGURATIONS
%----------------------------------------------------------------------------------------

\documentclass[letterpaper, 10pt]{article} 

\usepackage[english]{babel} % English language/hyphenation
\usepackage{graphicx}
\usepackage[lined,linesnumbered,commentsnumbered]{algorithm2e}
\usepackage{listings}
\usepackage{float}
\usepackage{fancyhdr} % Custom headers and footers
\pagestyle{fancyplain} % Makes all pages in the document conform to the custom headers and footers
\usepackage{lastpage}
\usepackage{url}
\usepackage{xcolor}
\usepackage{titlesec}
\usepackage{ulem}

% Stolen from https://www.overleaf.com/learn/latex/Code_listing 
\definecolor{codegreen}{rgb}{0,0.6,0}
\definecolor{codegray}{rgb}{0.5,0.5,0.5}
\definecolor{codepurple}{rgb}{0.58,0,0.82}
\definecolor{backcolour}{rgb}{0.95,0.95,0.92}

\lstdefinestyle{mystyle}{
    backgroundcolor=\color{backcolour},   
    commentstyle=\color{codegreen},
    keywordstyle=\color{magenta},
    numberstyle=\tiny\color{codegray},
    stringstyle=\color{codepurple},
    basicstyle=\ttfamily\footnotesize,
    breakatwhitespace=false,         
    breaklines=true,                 
    captionpos=b,                    
    keepspaces=true,                 
    numbers=left,                    
    numbersep=5pt,                  
    showspaces=false,                
    showstringspaces=false,
    showtabs=false,                  
    tabsize=2
}
\lstset{style=mystyle, language=c++}


\fancyhead{} % No page header - if you want one, create it in the same way as the footers below
\fancyfoot[L]{} % Empty left footer
\fancyfoot[C]{page \thepage\ of \pageref{LastPage}} % Page numbering for center footer
\fancyfoot[R]{}

\renewcommand{\headrulewidth}{0pt} % Remove header underlines
\renewcommand{\footrulewidth}{0pt} % Remove footer underlines
\setlength{\headheight}{13.6pt} % Customize the height of the header

%----------------------------------------------------------------------------------------
%	TITLE SECTION
%----------------------------------------------------------------------------------------

\newcommand{\horrule}[1]{\rule{\linewidth}{#1}} % Create horizontal rule command with 1 argument of height

\title{	
   \normalfont \normalsize 
   \textsc{CMPT 435 - Fall 2024 - Dr. Labouseur} \\[10pt] % Header stuff.
   \horrule{0.5pt} \\[0.25cm] 	% Top horizontal rule
   \huge Assignment Four -- Dynamic \& Greedy\\     	    % Assignment title
   \horrule{0.5pt} \\[0.25cm] 	% Bottom horizontal rule
}

\author{Ryan Munger \\ \normalsize Ryan.Munger1@marist.edu}

\date{\normalsize\today} 	% Today's date.

\begin{document}

\maketitle % Print the title

%----------------------------------------------------------------------------------------
%   CONTENT SECTION
%----------------------------------------------------------------------------------------

% - -- -  - -- -  - -- -  -
\section{Introduction}
\subsection{Goals}
Assignment 4 focuses on dynamic programming and greedy algorithms. First, I have to load in several weighted, directed graphs by parsing a file of instructions. Then, I must implement the Bellman-Ford dynamic programming algorithm	for Single Source Shortest Path (SSSP) in order to find the most efficient paths between nodes. Graphs aside, the greedy algorithm I need to implement is a version of the fractional knapsack problem. I need to read in a file that contains information about spices (price, quantity, etc.) and I must conduct a spice heist on Arrakis for each knapsack provided, maximizing my take.
\subsection{Write-up Format}
    In this report I will describe the logic being presented and the asymptotic running time of the algorithms implemented. Below the text explanation, relevant code will follow in C++. 
\subsection{Limerick of Luck}
\noindent
To maximally fill your knapsack \\
And remain unburdened with a fallback, \\
    \hspace*{1.5em}One must employ an algo of greed, \\
    \hspace*{1.5em}Your capacity will not exceed, \\
A heist for the ages, engraved on a plaque. \\
\newpage
\hrule
\vspace{.25cm}
Why wouldn't the greedy algorithm move?- \textit{Local maximums are everything to him...}\\
\hrule
\vspace{.25cm}

\section{Weighted Directed Graphs}
\subsection{Reading the Instructions}
I was provided a file of instructions to create weighted directed graphs such as \textit{add vertex} and \textit{add edge}. I was able to utilize most of the code from my previous assingment (with different regex) to create my graph. This time, I only had to create a linked object representation.  
\lstinputlisting[linerange={75-123}, firstnumber=75]{assignment4.cpp}

\subsection{Graph Object}
The graph object is even simpler than last time. Each graph has an ID and a map of linked vertex objects. This time, since my graph is weighted, I stored each edge in the neighbor list as a tuple. Adding vertices and edges is much simpler when we only have one representation to update! Since this is a directed graph, we only need to update a single object. I made a new graph display function to ensure the graphs were created correctly. 
\lstinputlisting[linerange={17-68}, firstnumber=17]{assignment4.cpp}

\subsection{SSSP}
I don't know yet! $O(Something bad, probably)$.
\begin{enumerate}
    \item Do something!
\end{enumerate}
\lstinputlisting[linerange={70-72}, firstnumber=70]{assignment4.cpp}
\hrule
\vspace{.25cm}
Why did they add a timer to chess? \textit{Mr. Dy Namic Algorithm...}\\
\hrule
\vspace{.5cm}

\subsection{Graph in Action}
Poop
\lstinputlisting[ linerange={3-20}, firstnumber=3]{cppout_for_writeup.txt}


\section{Greedy Knapsack}
\subsection{Gathering Information}
I gathered information about spices and knapsacks with regex in a similar fashion to my graphs. I stored my spices in a vector of Spice objects and my knapsack in a vector. I used float values for everything, as this is the \textit{fractional} knapsack problem, and there is no reason quantities and capacities cannot be decimal. 
\lstinputlisting[linerange={125-130}, firstnumber=125]{assignment4.cpp}
\lstinputlisting[linerange={183-223}, firstnumber=183]{assignment4.cpp}

\subsection{Organizing Spice}
To maximize take, we will examine the unit price of each spice (how much it is worth per quantity). To do this, we first sort the Spice list. I made a custom version of insertion sort to accomplish this. I put it in descending order. Since I used insertion sort, this action will take $O(n^2)$ time due to the nested loop. We can use a better sorting algorithm, such as merge or quick sort, to optimize this up to $O(log(n))$.
\lstinputlisting[linerange={140-155}, firstnumber=140]{assignment4.cpp}

\subsection{Maximizing Take}
I implemented a greedy algorithm. This class of algorithm takes locally optimal choices and hopes for a globally optimal solution. In this case, we will achieve a globally optimal solution by pillaging as much of the highest value spice we can fit, then the next, and so on. This algorithm will only take $O(n)$ time as it is simply a single traversal of the spice list. It is even less than a single traversal, as we can expect most knapsacks to fill up before we reach the end of our spice inventory list! Thus, fractional knapsack is a $O(nlog(n)$ algorithm if you count sorting the spice list, or $O(n)$ on its own.
\begin{enumerate}
    \item Examine the most valuable spice.
    \item If we have no more knapsack capacity, we are done.
    \item If we have more capacity than quantity of that spice, take everything! Record our scoops.
    \item If we have less capacity than the quantity of that spice, take as much as we can fit. Record scoops.
    \item Move to the next most valuable spice and repeat.
    \item Finally, report on our knapsack value and scoops taken.
\end{enumerate}
\lstinputlisting[linerange={157-181}, firstnumber=157]{assignment4.cpp}

\subsection{Greed in Action}
I have provided a few examples of Spices loaded, knapsacks created, and heists completed!
\lstinputlisting[linerange={98-130}, firstnumber=98]{cppout_for_writeup.txt}

\section{Conclusion}
Dynamic programming.... Greedy algorithms are much simpler than they seem. All that needs to be done is to take the greediest, most locally \& immediately optimal action. It is important to note though that while this did produce a globally optimal solution in this knapsack case, it does not always turn out this way, such as the cases of the 0-1 knapsack problem and traversing directed graphs!

\hrule
\vspace{.25cm}
Why did the greedy algorithm get full so quickly? \textit{It ate all the appetizers and spared no room!}
\vspace{.25cm}
\hrule
\vspace{.25cm}
Why did the dynamic algorithm score perfectly on its make-up test? \textit{It had all the answers saved from last time...}
\vspace{.25cm}
\hrule

\end{document}